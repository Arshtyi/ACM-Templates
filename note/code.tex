\documentclass{code}
\usepackage{code}
\title{Code Template}
\author{Arshtyi}
\date{\today}
\newcommand{\fzd}[1]{\ensuremath{O\left(#1\right)}}
\newcommand{\code}[1]{\lstinputlisting[language=CPP]{code/#1.cpp.txt}}
% \newcommand{\codeinline}[1]{\lstinline|#1|}
\begin{document}
% \maketitle
% \newpage
% \pagenumbering{Roman}
% \setcounter{page}{1}
% \pagestyle{plain}
% \tableofcontents
\setcounter{page}{1}
\pagenumbering{arabic}
\pagestyle{fancy}
\fancyhf{}
\fancyhead[L]{\leftmark}
\fancyhead[R]{\thepage}
\part{模板}
\chapter{main}
\thispagestyle{fancy}
\section{main}
\code{main}
\newpage
\chapter{排序}
\thispagestyle{fancy}
\section{快排}
时间复杂度:\fzd{n \log n}.
\code{QuickSort}
\newpage
\chapter{查询}
\thispagestyle{fancy}
\section{二分查找}
时间复杂度:\fzd{\log n}.
\code{BinarySearch}
\section{ST表}
时间复杂度:\fzd{1}.
\code{SparseTable}
\newpage
\chapter{搜索}
\thispagestyle{fancy}
\section{DFS}
时间复杂度:\fzd{V + E}.
\code{DFS}
\section{BFS}
时间复杂度:\fzd{V + E}.
\code{BFS}
\newpage
\chapter{二叉树}
\thispagestyle{fancy}
\section{建树}
\subsection{根据节点信息建树}
根据DFS得到深度.
\code{BuildTreeByDFS}
\section{遍历}
\subsection{层次遍历}
每次取出队列首,加入其子结点到队列尾.时间复杂度:\fzd{n}.
\code{LevelOrderTraversal}
\subsection{顺序遍历}
前序、中序、后序本质上都是DFS.
\code{OrderTraversal}
可以根据两种顺序遍历给出第三种遍历的结果.
\code{OrderTraversalByTwoOrders}
\newpage
\chapter{图论}
\thispagestyle{fancy}
\section{拓扑排序}
时间复杂度:\fzd{V + E}.
\code{TopologicalSort}
\newpage
\chapter{数论}
\thispagestyle{fancy}
\section{素数筛}
埃氏筛时间复杂度为\fzd{n \log \log n}.
\code{SieveOfEratosthenes}
\newpage
\chapter{RHS}
\thispagestyle{fancy}
\section{快速幂}
时间复杂度:\fzd{\log n}.
\code{FastPower}
\section{欧几里得算法}
时间复杂度:\fzd{\log n}.
\code{EuclideanAlgorithm}
\section{乘法逆元}
\code{ModularInverse}
\section{链表}
\subsection{单向链表}
\code{SingleLinkedList}
\subsection{双向链表}
\code{DoubleLinkedList}
\subsection{循环单向链表}
\code{CircularSingleLinkedList}
\section{并查集}
路径压缩的时间复杂度:\fzd{\log n}.
\code{UnionFind}
\section{高精度数类}
\code{BigInteger}
\part{Hints}
\chapter{位运算}
\thispagestyle{fancy}
\section{异或}
\begin{enumerate}[label=\arabic*)]
    \item $\displaystyle\bigoplus _{i=1}^{n} i =
              \begin{cases}
                  n     & \text{if } n \equiv 0 \pmod{4} \\
                  1     & \text{if } n \equiv 1 \pmod{4} \\
                  n + 1 & \text{if } n \equiv 2 \pmod{4} \\
                  0     & \text{if } n \equiv 3 \pmod{4}
              \end{cases}$
    \item $\displaystyle\bigoplus_{i=1}^{n} x =
              \begin{cases}
                  0 & \text{if $n$ is even} \\
                  x & \text{if $n$ is odd}
              \end{cases}$
    \item $x \oplus 0 = x$
\end{enumerate}
\chapter{数论}
\thispagestyle{fancy}
\section{除数函数}
记$\displaystyle a=p_{1}^{\alpha_{1}}p_{2}^{\alpha_{2}}\cdots p_{s}^{\alpha_{s}}$,则除数函数$\tau \left(a\right)$表示$a$所有正约数的个数,表达为\begin{align*}
    \tau \left(a\right) & =\prod_{i=1}^{s}\left(\alpha_{i}+1\right)=\prod_{i=1}^{s}\tau \left(p_{i}^{\alpha_{i}}\right) \\
                        & =\sum_{d\mid a} 1=\sum_{i=1}^n\lfloor\frac{n}{i}\rfloor
\end{align*}
除数和函数$\sigma \left(a\right)$表示$a$所有正约数的和,表达为
\begin{align*}
    \sigma \left(a\right) & =\prod_{i=1}^{s}\frac{p_{i}^{\alpha_{i}+1}-1}{p_{i}-1}=\prod_{i=1}^{s}\sigma \left(p_{i}^{\alpha_{i}}\right)
\end{align*}
\section{Fibonacci数列}
\begin{align*}
    F_{n} & = \begin{cases}
                  0                 & n = 0 \\
                  1                 & n = 1 \\
                  f_{n-1} + f_{n-2} & n > 1
              \end{cases}                                                                                 \\
          & =\frac{1}{\sqrt{5}}\left[\left(\frac{1+\sqrt{5}}{2}\right)^{n}+\left(\frac{1-\sqrt{5}}{2}\right)^{n}\right]
\end{align*}
\subsection{基础性质}
\[
    F_{m+n}=F_mF_{n+1} + F_{m-1}F_n\Longrightarrow\begin{cases}
        F_{2n+1} = F_n^2 + F_{n+1}^2 \\
        F_{2n+2} = F_{n+2}^2-F_n^2
    \end{cases}
\]
\subsection{Cassini恒等式}
\[
    F_{n-1}F_{n+1} - F_{n}^{2} = (-1)^{n}
\]
\section{RHS}
\subsection{所有正整数的平方差}
所有正整数的平方差的集合为所有大于等于$3$的奇数和所有大于等于$8$的$4$的倍数的集合也就是
\begin{align*}
    \left\{y^2-x^2\mid\forall 1\leqslant x<y\in \mathbb{Z}^+\right\}=\left\{x\mid x=2k+1,k\geqslant 1\vee x=4k,k\geqslant 2\right\}
\end{align*}
\chapter{组合数学}
\thispagestyle{fancy}
\section{组合数的递推计算}
组合数的递推公式为
\begin{align*}
    C_{n}^{m} = C_{n-1}^{m-1} + C_{n-1}^{m}
\end{align*}
\section{格点图中的矩形个数}
一个$N\times M$的格点图中,(边平行于坐标轴的)矩形的总个数为
\[
    C_{N+1}^{2} \cdot C_{M+1}^{2} = \frac{MN\left(M+1\right)\left(N+1\right)}{4}
\]特别的,正方形的个数为\[
    \sum_{i=1}^{\min\left(M,N\right)} \left(M-i+1\right)\left(N-i+1\right)
\]
\chapter{背包问题}
\thispagestyle{fancy}
\section{简单01背包}
给出背包容量$V$和$n$件物品,每件物品的体积为$v_i$,价值为$w_i$,求在不超过背包容量的前提下,所能获得的最大价值.定义$DP[i][j]$表示前$i$件物品在容量为$j$的背包中所能获得的最大价值,则状态转移方程为
\[
    DP[i][j] = \max\{DP[i-1][j], DP[i-1][j-v_i]+w_i\}
\]
边界条件为$DP[0][j]=0$,最终答案为$DP[n][V]$.
\chapter{构造问题}
\thispagestyle{fancy}
\section{幻方的构造}
幻方的构造根据阶数$n\in\mathbb{N}^+$分为三种
\subsection{奇数阶幻方}
对于$n=2k+1$的奇数阶幻方,构造方法为先把$1$放在第一行中间位置,接着对于每一个数$i\in\left[2,n\times n\right]$将其尽可能填在$i-1$的右上方,如果该位置超出顶角或者已经被填充,则将其放在$i-1$的正下方,直到填满所有位置
\incfig{0.4}{MagicSquare1}
\subsection{双偶阶幻方}
对于$n=4k$的双偶阶幻方,构造方法为先将其划分为一个个$4\times 4$的子幻方,然后从第$1$行第$1$列开始,按照从左到右、从上到下的顺序依次填充$1,2,\cdots,n^2$并且每个子幻方的对角线上不填充,接着从第$n$行第$n$列开始,按照从右到左、从下到上依次填充$1,2,\cdots,n^2$填充,这样就得到了一个双偶阶幻方.
\incfig{0.9}{MagicSquare2}
\subsection{单偶阶幻方}
对于$n=4k+2$的单偶阶幻方,构造方法为从上到下、从左到右分为四个象限$A$、$B$、$C$、$D$,每个象限内部按照奇数阶幻方填充,令$\displaystyle m=\frac{n-2}{4}$,以$A$象限的中心一格为第一格,取定包含第一格在内的共$m$格,然后取定$A$象限的不包括中心行、列在内的左边$m$格,这些与$C$象限对换.然后取定$B$象限包括中心列在内的左边$m$列,这些与$D$象限对换.
\end{document}
