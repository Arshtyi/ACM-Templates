\documentclass{code}
\usepackage{code}
\title{\Huge{Code Template}}
\author{\huge{Arshtyi}}
\date{\today}
\newcommand{\fzd}[1]{\ensuremath{O\left(#1\right)}}
\newcommand{\code}[1]{\lstinputlisting[language=CPP]{code/#1.cpp.txt}}
% \newcommand{\codeinline}[1]{\lstinline|#1|}
\begin{document}
\maketitle
\newpage
\pagenumbering{Roman}
\setcounter{page}{1}
\pagestyle{plain}
\tableofcontents
\newpage
\setcounter{page}{1}
\pagenumbering{arabic}
\pagestyle{fancy}
\fancyhf{}
\fancyhead[L]{\leftmark}
\fancyhead[R]{\thepage}
\section{I/O}
\subsection{开始坐牢}
\code{main}
\newpage
\section{排序}
\subsection{快速排序}
时间复杂度:\fzd{n \log n},空间复杂度:\fzd{n}.
\code{QuickSort}
\newpage
\section{查询}
\subsection{二分查找}
时间复杂度:\fzd{\log n},空间复杂度:\fzd{1}.
\code{BinarySearch}
\newpage
\section{搜索}
\subsection{\texorpdfstring{$DFS$}{DFS}}
时间复杂度:\fzd{V + E},空间复杂度:\fzd{V}.
\code{DFS}
\subsection{\texorpdfstring{$BFS$}{BFS}}
时间复杂度:\fzd{V + E},空间复杂度:\fzd{V}.
\code{BFS}
\newpage
\section{二叉树}
\subsection{建树}
\subsection{根据节点信息建树}
根据$DFS$得到深度.
\code{BuildTreeByDFS}
\subsection{遍历}
\subsubsection{层次遍历}
每次取出队列首,加入其子结点到队列尾.时间复杂度:\fzd{n},空间复杂度:\fzd{n}.
\code{LevelOrderTraversal}
\subsubsection{顺序遍历}
前序、中序、后序本质上都是$DFS$.
\code{OrderTraversal}
\newpage
\section{图}
\subsection{拓扑排序}
时间复杂度:\fzd{V + E},空间复杂度:\fzd{V}.
\code{TopologicalSort}
\newpage
\section{数论}
\subsection{素数筛}
埃氏筛时间复杂度为\fzd{n \log \log n}.
\code{SieveOfEratosthenes}
\newpage
\section{RHS}
\subsection{快速幂}
时间复杂度:\fzd{\log n},空间复杂度:\fzd{1}.
\code{FastPower}
\subsection{链表}
\subsubsection{单向链表}
\code{SingleLinkedList}
\subsubsection{双向链表}
\code{DoubleLinkedList}
\subsubsection{循环单向链表}
\code{CircularSingleLinkedList}
\subsection{并查集}
路径压缩的时间复杂度:\fzd{\log n}.
\code{UnionFind}
\subsection{高精度数类}
\code{BigInteger}
\newpage
\section{Hints}
\subsection{位运算}
\subsubsection{异或}
\begin{enumerate}[label=\arabic*)]
    \item $\displaystyle\bigoplus_{i=1}^{n} i =
        \begin{cases}
            n      & \text{if } n \equiv 0 \pmod{4} \\
            1      & \text{if } n \equiv 1 \pmod{4} \\
            n + 1  & \text{if } n \equiv 2 \pmod{4} \\
            0      & \text{if } n \equiv 3 \pmod{4}
        \end{cases}$
    \item $\displaystyle\bigoplus_{i=1}^{n} x =
        \begin{cases}
            0 & \text{if $n$ is even} \\
            x & \text{if $n$ is odd}
        \end{cases}$
    \item $x \oplus 0 = x$
\end{enumerate}
\subsection{数论}
\subsubsection{所有正整数的平方差}
所有正整数的平方差的集合为所有大于等于$3$的奇数和所有大于等于$8$的$4$的倍数的集合也就是
\begin{align*}
    \left\{y^2-x^2\mid\forall 1\leqslant x<y\in \mathbb{Z}^+\right\}=\left\{x\mid x=2k+1,k\geqslant 1\vee x=4k,k\geqslant 2\right\}
\end{align*}
\subsubsection{除数函数}
记$\displaystyle a=p_{1}^{\alpha_{1}}p_{2}^{\alpha_{2}}\cdots p_{s}^{\alpha_{s}}$,则除数函数$\tau \left(a\right)$表示$a$所有正约数的个数,表达为\begin{align*}
    \tau \left(a\right) &=\prod_{i=1}^{s}\left(\alpha_{i}+1\right)\\
    &=\prod_{i=1}^{s}\tau \left(p_{i}^{\alpha_{i}}\right)
\end{align*}
除数和函数$\sigma \left(a\right)$表示$a$所有正约数的和,表达为
\begin{align*}
    \sigma \left(a\right) &=\prod_{i=1}^{s}\frac{p_{i}^{\alpha_{i}+1}-1}{p_{i}-1}\\
    &=\prod_{i=1}^{s}\sigma \left(p_{i}^{\alpha_{i}}\right)
\end{align*}
\subsection{组合数学}
\subsubsection{组合数的递推计算}
组合数的递推公式为
\begin{align*}
    C_{n}^{m} = C_{n-1}^{m-1} + C_{n-1}^{m}
\end{align*}
\end{document}
